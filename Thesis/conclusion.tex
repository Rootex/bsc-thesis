\chapter{Conclusion}
In this thesis, I researched on modular development approach and provided information on the concept. The approach provides a solution to the problem of developing large systems either in teams or as an individual. Following the approach, there are two important development phases which I took on. 

{\bf Modular development} as discussed is nothing more than a development approach that lays its basis on separation of concerns, it encourages the idea of dividing a system into smaller functional modules, developing those modules independent of others and leaving no dependency between two modules unless a module needs another to achieve its goal. This approach also puts emphasis on design first, {\bf modular design}, then implementation, {\bf modular programming}.

Designing a system before implementation is mandatory in this development approach, so i provided the structure of the system, an architectural diagram and two UML models of the system which was used in the implementation phase. This models are the {\bf Use Case diagram} which shows the different use case i.e ways users interact with the system and a {\bf Class Diagram} which shows the different classes of the system and the relationship between them.

The {\bf eye droid} system is an android application which was developed for android devices with the aid of the android development kit. The languages used to implement this system are the Java  programming language which is android's native programming language and Python programming language (Kivy's back-end language). Kivy is an open source development framework which enables rapid application development and can be compiled for multiple platforms.    

The application system is a surveillance system which allows users to use their android devices to monitor a particular environment. This application utilizes the android camera component to detect faces and notifies the user via both SMS and EMAIL in real time. Also the application keeps a log of the events that happened during the camera preview and saves the faces that where detected on the device's storage. The log can be reviewed by the user with the review activity and the faces can be viewed using the viewer module.

From this thesis, I can deduce that it is much more easier and flexible to develop complex software systems by using modular development approach and this concept can be used in android development. I used the feature of packages provided by Java  to package modules and this modules can be compiled individually as .jar files and imported to any project that its needed. This is not the only type of modules, complete android applications themselves can be modules, and can be integrated into a software system by the use of intents, illustration of that is the viewer module which i developed as an individual application and the way its integrated can be seen in the main activity.

There is a lot of extension that can be done to this application, in the future, i intend to extend this system to provide the support for motion detection, face tracking and go to the extent of developing an easier way modules can be integrated into android system. I also intend on researching and experimenting different ways modularity can be achieved in various platforms.

